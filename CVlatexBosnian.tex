\documentclass[11pt, a4paper]{moderncv}
\label{key}
\usepackage[T1]{fontenc}
\usepackage[utf8]{inputenc}
\usepackage[left=2.4cm, right=1.0cm, top=1.7cm, bottom=2.0cm]{geometry}
\usepackage{csquotes}
\usepackage[utf8]{inputenc}
\usepackage[T1]{fontenc}
\usepackage{currvita}
\usepackage{graphicx}
\usepackage{hyperref}
\graphicspath{ {./images/} }

\moderncvtheme[black]{classic}

 %Die persönlichen Daten:
\name{Hana}{Kovačević}
%\title{Curriculum Vitae}%
\address{Olimpijska 40}{71000 Sarajevo}
\phone[mobile]{+387\,62\,495\,748}
\email{hanakovacevic@protonmail.com}
\social[linkedin][www.linkedin.com/in/hana-kova\%C4\%8Devi\%C4\%87-577719128/]{Hana Kovačevi\'{c}}
\social[github][www.github.com/hanascodes]{hanascodes}
%\social[reddit]{https://www.reddit.com/user/theredhana/}%
\photo[3cm]{wasped3}




\begin{document}
  \makecvtitle

  \section{Lični podaci}
  \cvline{Dan rođenja:}{16.06.1997. [22]}
  \cvline{Mjesto rođenja:}{Travnik, Bosna i Hercegovina}
  \cvline{Broj telefona}{+38762495748}


  \cvline{Web stranica \textit{under construction}}{\href{https://hanaisnotonline.online}{hanaisonline}}

  \section{Obrazovanje}
  \cventry{2019--danas}{IT stručni studij}{Prirodno-matematički fakultet}{}{}{Fokus na kibernetičku sigurnost}
  \cventry{2012--2016}{Srednja škola}{Gimnazija Dobrinja}{}{}{Usmjerenje: Lingivističko \\}

  \section{Certifikati}
  \cventry{Septembar 2018}{Hacker's BootCAMP}{Cyber Academy (Sentry Cyber Security)}{}{}{
  \enquote{Hands-on Ethical Hacking}: building security, Python programming and Linux for hackers}
  \cventry{Januar 2018}{Švedski jezik - A1 nivo}{Schoolplus}{}{}{}
  \cventry{November 2018}{Hour of Code}{code.org}{}{}{}

  \section{Radno iskustvo}
  \cventry{Maj-Oktobar 2018}{ESL učiteljica za mlađu djecu}{Panda ABC}{}{}{}
  \cventry{Novembar 2017\newline-Februar 2018}{ESL učiteljica za odrasle}{Bibo Global Opportunity}{}{}{}
  \cventry{Oktobar 2019\newline-Februar 2020}{Risk Assurance Associate}{PricewaterhouseCoopers}{}{}{IT audit}

  \section{Volontiranje}
  \cventry{September 2018}{Volonterka na izbornoj kampanji}{Naša Stranka}{}{}{Door-to-door kampanja}
  \cventry{Novembar 2016}{Gostoprimstvo}{Pravo Ljudski Film Festival}{}{}{}
  \cventry{Juni 2014}{Volonterka}{Sarajevo Peace Event}{}{}{}
  \cventry{Oct-Dec 2018}{Volonterka}{Aid Brigade}{}{}{}
  \cventry{Novembar 2018}{Event organizer}{Christmas Miracle (IYF)}{}{}{}

\pagebreak

  \section{IT vještine}
  \cvline{Operativni sistemi}{Windows, Linux (Kali, Ubuntu, Manjaro)}
  \cvline{Programski jezici}{C++, Python, Latex}
  \cvline{Alati za penetracijska testiranja}{nmap, Burpsuite, Nessus}
  \cvline{Microsoft 365}{MS Word, Powerpoint, Excel}
  \cvline{Libreoffice}{Writer, Impress, Calc}


  \section{Jezici}
  \cvdoubleitem{Bosanski}{Izvorni govornik}{Njemački}{Srednji nivo (B1)}
  \cvdoubleitem{Engleski}{Tečno (C2)}{Švedski}{Početni nivo (A1)}

  \section{Hobiji i interesi}
  \cvitem{}{Powerlifting}
  \cvitem{}{Muzika (Sviranje gitare, pjevanje)}
  \cvitem{}{Trčanje}
  \cvitem{}{Digitalna umjetnost}
  \cvitem{}{Gaming}

  \section{Radno iskustvo - nastavak}

  \cventry{Panda ABC}{ESL učiteljica}{Moje primarne odgovornosti su bile da podučavam sa velikom dozom entuzijazma, za razliku od tradicionalnog pristupa podučavanju koji se oslanja na rigorozno testiranje i strogo podučavanje. Organizirala sam časove, pripremala lekcije za đake i pisala feedback nakon svakog časa. Časovi su se održavali preko interneta kroz program koji je sama firma razvila. Veličina razreda je rangirala od 1-6, a dob učenika i učenica 1-16}{}{}{}

  \cventry{Bibo Global Opportunity}{ESL teacher}{Kao predavačica u BGO, pretežno sam bila neka vrsta online prijateljima velikom broju ljudi iz cijelog svijeta (primarno Japanu). Većini sam podučavala razgovorni engleski. Posao je bio super, jer mi je dao priliku da dobijem sliku života van Bosne i Hercegovine}{}{}{}

  \cventry{PwC}{Risk Assurance Associate}{Kao praktikantkinja u PwC-u, radila sam u Risk Assurance odjelu kao revizorka informacijskih sistema.\newline
Moje odgovornosti su bile sljedeće:\newline
1. Asistiranje mentorici i svojoj team leaderki u raznim zadacima;\newline
2. Rješavanje trivialnih i kompleksnih kompjuterskih problema;\newline
3. Razumijevanje poslovnih procesa;\newline
4. Obrada i analiza klijentskih podataka zadobljavanjem i provjeravanje revizorskih podataka, priprema i uređivanje konačnih izvještaja;\newline
5. Održavanje aktivne komunikacije sa klijentima, kolegama i kolegicama;\newline
6. Prioritizacija ciljeva;\newline
7. Obavljanje zadaka vezanih za kibernetičku sigurnost (testiranje ranjivosti na mrežama, testiranje password politika, provjere fizičke sigurnosti, itd.)\newline}{}{}{}

  \vfill

  Činjenica: ovaj CV je napisan u Latex-u. Izvorni kod se može naći na mom Githubu.

\end{document}
